\documentclass[rapport.tex]{subfiles}
\begin{document}
\chapter*{\uppercase{Discussion}}
\addcontentsline{toc}{chapter}{DISCUSSION}
En rappel, l’objectif initial du projet était la production d’une application sur tablette Apple permettant de scanner un objet de l’environnement réel à l’aide du Structure Sensor et le présenter sur une surface plane en réalité augmentée à l’aide de l’ARKit. Le produit satisfait cette attente à la fin de la période de développement soit la fin de la troisième itération. En effet, les deux interfaces nommées sont utilisées. De plus, le cas d’utilisation impliqué par l’objectif est tout à fait possible.
\section*{Comparatif du produit planifié et actuel}
\addcontentsline{toc}{section}{Comparatif du produit planifié et actuel}
Au début du processus de conception, l’équipe a identifié une cible pour le produit final. Maintenant la période de développement finalisé, l’état actuel de l’application doit être comparé avec la cible. Les exigences, les risques et les hypothèses identifiés seront alors revisités.
\subsection*{Exigences}
\addcontentsline{toc}{subsection}{Exigences}
Comme on peut le voir en Annexe I, le statut des exigences de l’objectif principal a pour la plupart été complété. Les écarts entre celles planifiées et celles réellement complétées sont le résultat de décisions prises lors du processus de conception. Une entente avec la résolution à entreprendre a été discutée avec tous les membres de l’équipe.
\par
Parmi ces écarts, on trouve l’exigence ST11 portant le nom ouverture sans caméra qui indiquait que l’application devrait pouvoir être ouverte sans accès à la caméra. Cette exigence a été annulée puisque la caméra est à la base des fonctionnalités de l’application. Donc, la gestion de l’exception semblait inutile. Surtout étant donné, le travail supplémentaire que celle-ci impliquait quand le stade de déploiement ciblé est un prototype.
\par
L’exigence ST22 nommée Interface descriptive de modèle a aussi été annulée. Cette exigence impliquait la création d’une nouvelle interface dans l’application pour décrire le modèle. Celle-ci demandait donc un travail supplémentaire de conception, l’ajout de composante de navigation et la recherche de statistiques de fichiers internes à une application. L’utilité de cette interface ne paraissait pas suffisante pour la prioriser par rapport à d’autres, plus importantes pour le bon fonctionnement.
\par
L’application actuelle consiste pour la plus grande partie en l’intégration de plusieurs interfaces de programmation dans un même produit. Les différentes interfaces sont testées par les institutions les déployant et publiant. Ce qui limite grandement le code qui n’est pas testé dans notre application. C’est pourquoi l’exigence code testable avec l’identifiant ST24 a aussi été mis de côté. Le niveau de déploiement prototype et le nombre de travail nécessaires à la création de tests étaient trop élevés.
\par
Deux exigences, ST27 et ST28, qui consistent en l’exportation des fichiers créés lors des scans et des modifications dans l’interface de réalité augmentée. Celles-ci sont considérées comme partiellement fonctionnelles puisque lors de la connexion sur l’outil de développement XCode on a accès à ces fichiers. Donc, le propriétaire du prototype peut les exporter.
\par
Les exigences ST33 à ST40 ciblaient l’objectif accessoire d’avoir un squelette de nœuds dans le modèle pour pouvoir l’animer. Cet objectif n’a pas pu être complété dans la période disponible pour l’exécution du projet. Donc, ces exigences ont été reportées à une poursuite possible du projet.
\subsection*{Risques}
\addcontentsline{toc}{subsection}{Risques}
Les risques identifiés dans la planification ont bien été des causes de problèmes lors du développement. Les impacts et les probabilités prévues pour chacun respectaient bien l’échelle réelle. Les dépendances de technologiques, le manque de documentation et le manque de leadership dans le projet ont introduit des ralentissements dans la production de l’application.
\par
Le plus grand défi fut en effet les difficultés d’implémentation des interfaces de programmation. Le changement de l’interface de programmation du Structure Sensor impliqué par le changement de langage d’Objective-C à Swift. Un grand nombre de temps a été mis sur le transfert d’un format à l’autre.
\subsection*{Hypothèses}
Comme indiqué dans la section précédente, la difficulté d’implémenter les interfaces de programmation a été le plus grand défi du projet. Ce qui confirme la première hypothèse HYP1. Une des raisons pour lequel l’implémentation à l’aide de l’ARKit au lieu de tout faire dans le contexte Structure Sensor était en réponse à une très petite communauté de développement sur le périphérique. Alors que le nombre de développeurs utilisant le ARKit était plus intéressant pour aller chercher de l’aide chez la communauté. Cette hypothèse, HYP4, a été en effet confirmée. En effet, 13 millions d'applications ARKit ont été téléchargées entre octobre 2017 et mars 2018. \citep*{remi}
\par
Les deux autres hypothèses n’ont pas pu être confirmées ou invalidées puisqu’elle adressé l’objectif accessoire du squelette humain. Par contre, la vraissembalité de celles-ci n’a pas changé aux yeux de l’équipe de travail. Donc, la validation peut être reportée à la poursuite possible du projet.
\addcontentsline{toc}{subsection}{Hypothèses}
\section*{Améliorations possibles}
\addcontentsline{toc}{section}{Améliorations possibles}
Étant donné la durée restreinte de ce projet, il est certain que certaines étapes de peaufinage ont été abandonnées afin de laisser place à un produit fonctionnel d’un bout à l’autre. Comme mentionné plus tôt, beaucoup d’inconnus technologiques ont causés des réalignements dans la portée du projet. Dans cette section, chaque amélioration possible à notre produit courant sera énumérée et expliquée en détails
\subsection*{Manipulation des nouveaux objets}
\addcontentsline{toc}{subsection}{Manipulation des nouveaux objets}
Le fonctionnement de bout en bout de l’ajout d’un nouveau modèle scanné à l'environnement AR a seulement été un succès lors de la dernière itération. Pour cette raison il a été impossible de rendre ce phénomène le plus élégant possible. Dans son état actuel, un nouveau scan produit un fichier .ZIP qui est stocké à la racine du projet afin que les deux parties du logiciel (AR et Scanner) puissent y accéder. Par la suite, lors de la sélection de ce modèle en particulier dans la liste des objets disponible, ce fichier est décompressé dans un dossier temporaire et un traitement numérique est fait sur cet objet afin de le convertir en objet utilisable par le ARKit de Apple. Sans avoir eu de contrainte de temps, il aurait été beaucoup plus élégant de faire tout ce traitement numérique une seule fois lors de l’enregistrement d’un nouveau scan à notre application. Cette modification sauverait du temps de calcul CPU et mémoire de l’appareil sur lequel ce logiciel roulera dans le futur. Cette erreur de conception a été insérée au produit final lorsque des problèmes de permissions d’accès aux fichiers sont survenus et que l’équipe a voulu contourner un problème de façon à pouvoir continuer le développement sans impacter les autres composantes.
\subsection*{Interface utilisateur du scanneur}
\addcontentsline{toc}{subsection}{Interface utilisateur du scanneur}
Comme mentionné dans les sections précédentes, des changements majeurs ont dû être effectués sur le code d’appel à l’API du Structure Sensor, car ses classes d’interfaces fournies étaient toutes conçues pour de vieilles versions de Swift. En se basant sur les projets démonstratifs de l’API, l’équipe pensait être capable de faire beaucoup de réutilisation, mais étant donné que de la réingénierie a été nécessaire dans les appels à l’API d'interface du matériel et du traitement des données, tout le côté visuel de l’application de démonstration a été conservé. Le but initial était de refaire une interface visuelle pour interagir avec le code modifié, mais le manque de temps causé par le manque de documentation pour les récentes versions de Swift ont causé du retard dans la sous-équipe responsable de la partie scanneur. Le problème c’est que l’on pourrait croire que l’application est fournie dans l’API a été insérer dans notre produit sans aucune modification, due à son interface presque identique. Cependant, le code qui roule derrière est très différent et même que la plupart des interactions fournies par l’interface de l’API ne sont plus fonctionnelles, car nos besoins n'incluent pas une configuration en temps réel du scanneur. L’amélioration, ici, serait de créer un nouveau Storyboard, de la connecter sur notre code actuel et ainsi pouvoir se débarrasser de cette interface utilisateur générique.
\section*{Prochaines étapes}
\addcontentsline{toc}{section}{Prochaines étapes}
L’application produite dans le cadre du projet n’étant qu’un prototype et ayant plus d’objectifs implique une poursuite du travail possible et intéressante.
\subsection*{Squelette de noeuds}
\addcontentsline{toc}{subsection}{Squelette de noeuds}
Le projet consistait en deux objectifs, le premier la création de l’application intégrant les deux technologies. Celui-ci a été complété. Toutefois, le deuxième qui lui était accessoire, la création d’un squelette de nœuds dans le modèle humain, n’a pas pu être complété. Les prochaines étapes seraient donc de compléter celui-ci.
\par
Le développement d’un tel prototype pourrait permettre d’évaluer les hypothèses, HYP2 et HYP3 qui ont été identifiés au début du projet. Ces hypothèses permettront d’adresser les différentes inquiétudes par rapport à la performance d’une application comme celle produite dans le projet dans un contexte réel.
\subsection*{Publication d'une application}
\addcontentsline{toc}{subsection}{Publication d'une application}
Le niveau de déploiement du produit est actuellement seulement un prototype. Donc une fois les deux objectifs complétés, il serait intéressant de déployer une version publique de l’application sur le marché d’Apple. Toutefois, certaines tâches devront être accomplies avant d’être prêtes à une publication. Des tests devront être ajoutés à l’application. L’export des modèles scanné devra être envoyé sur iCloud pour permettre le transfert entre les appareils de l’usager. Une validation des droits associés aux périphériques produit par Occipital, la compagnie propriétaire du Structure Sensor, devra être faite.
\end{document}
