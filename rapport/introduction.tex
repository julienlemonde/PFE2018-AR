\documentclass[rapport.tex]{subfiles}
\begin{document}
\chapter*{\uppercase{Introduction}}
\addcontentsline{toc}{chapter}{INTRODUCTION}
La performance des caméras des appareils mobiles est en constante croissance depuis quelques années. Ce qui permet d’envisager l’usage de ces appareils dans des nouveaux domaines d’activités. Un de ces domaines est la réalité augmentée, où la superposition d’éléments virtuels sur une capture de l’environnement réel en temps réel. Une démocratisation du développement d’application utilisant cette technologie s’est faite au cours de l’année précédente, 2017. En effet, la nouvelle interface de programmation applicative ARKit conçue par Apple offre un environnement de développement simplifié au développeur d’applications \citep*{StattN2017}. Toutefois, ces technologies permettent l’interaction avec des modèles générés de façon synthétique. Un défi qui n’a pas encore de solution reconnue est l’usage de modèles provenant de l’environnement de l’utilisateur dans la réalité augmentée. Certaines technologies offrent des pistes de solution, mais leur taux d’adoption et le niveau de compatibilité avec les autres outils de l’environnement de développement Apple est limité.
\par
Une des technologies les plus intéressantes est le Structure Sensor\citep*{molitchHou01}. Celle-ci consiste en un périphérique qui doit être attaché à l’appareil. Le périphérique ajoute une caméra et un senseur 3D. Avec l’usage de la caméra de l’appareil, celui de l’appareil on peut avoir une meilleure compréhension de l’environnement 3D. L’outil vient avec une trousse de développement logiciel permettant de prendre les coordonnées d’un objet sous quatre dimensions, x, y, z et la couleur. Toutefois cette solution n’est pas très connue et a un nombre de ressources limitées.
\par
Les deux technologies décrites plus haut apportent chacune des possibilités très intéressantes pour le développement d’application de réalité augmentée. L’usage des fonctionnalités d’une dans l’autre serait très intéressant. Toutefois au moment où l’équipe a débuté le projet, il n’y a pas d’interface une telle combinaison. Les deux interfaces utilisent la caméra, par contre chacun utilise un utilitaire différent pour le contrôle de celle-ci. Les deux utilisent des modèles numériques 3D, mais chacun a choisi une norme différente. En effet, le Structure Sensor permet l’extraction en Modèle I/O. Alors que le ARKit utilise les modèles de type scène qui est un format propriétaire à Apple. Le principal défi du projet consistera à la combinaison de plusieurs interfaces de programmation pour créer un outil fonctionnel.
\par
Comme décrit dans la section précédente, le principal objectif du projet est la création d’une interface entre deux technologies mobiles, soit un périphérique d’acquisition de modèle 3D, le Structure Sensor et la trousse de développement de réalité augmentée d’Apple. Autrement dit, l’équipe s’attend en fin de session d’avoir produit une application mobile permettant la prise de modèle à l’aide du senseur. L’application permettra d’ajouter un modèle extrait sur une surface plane. L’application permet cette chaîne d’action avec une expérience utilisateur agréable et naturelle à l’utilisateur.
\par
L’application décrite ci-haut représente les attentes pessimistes par rapport aux résultats finaux. Toutefois, une telle application apporte un éventail de possibilités sur la manipulation de l’environnement augmentée. Par exemple, une des retombées envisageables et commercialisables est la numérisation de catalogue de magasin comme Ikea. L’usager peut donc ajouter un item dans une pièce de sa maison. Si plusieurs couleurs sont disponibles pour un modèle, chacune des variantes sera disponible à la modélisation.
\par
Une autre ouverture technologique, qu’une telle application apporterait, est dans un contexte d’animation 3D. En effet, la création d’un modèle permettrait l’ajout d’un squelette dans celui-ci. Une fois le squelette créé l’animation du modèle pourrait être fait. Cette possibilité est particulièrement intéressante pour l’équipe responsable du projet. Dans l’éventualité où la vision de base du projet sera complétée avant la fin de la session d’Hiver 2018, l’équipe compte se concentrer sur la création de ce squelette et possiblement la manipulation de celui-ci. La création de positionnement clé des nœuds du squelette pourra être enregistrée afin de permettre l’animation. L’animation libre est un trop grand défi étant donné les contraintes de temps.
\par
L’objectif principal, une application sur tablette Apple permettant de scanner un objet de l’environnement réel à l’aide du Structure Sensor et le présenter sur une surface plane en réalité augmentée à l’aide de l’ARKit, semble réaliste pour l’équipe de développement. Toutefois, le défi reste important étant donné le manque de cohérence entre les différentes technologies. Le manque de documentation et de support pour le Structure Sensor risque d’amener un ensemble de défi supplémentaire.
\par
Par contre, l’équipe de développement considère le temps disponible à la résolution d’un tel enjeu adéquat. L’expérience en recherche d’information et apprentissage autonome acquise au cours des formations et stages des membres de l’équipe laisse envisager un succès.
\par
Pour ce qui est en la complétion des objectifs accessoire, les limitations de temps rendent l’estimation de succès plus difficile. Par contre, il est clair que la réalisation de ceux est apporterait un plus considérable à l’expérience de chacun des membres. La motivation de l’équipe face à l’enjeu améliore les probabilités de terminer ces objectifs. Ce rapport consiste donc en un suivi des étapes qui permettront de valider ou invalider les hypothèses décrites si haut. Afin de résoudre un tel défi, l’équipe a dû se fier sur les documentations disponibles en lien avec les technologies ciblées. Un résumé de celles qui ont été utiles à la résolution du problème sera présenté.
\par
Par la suite, une présentation de la méthodologie utilisée par l’équipe sera décrite. Les étapes ayant permis la conception de la solution seront par la suite présentées. Une analyse sur le produit obtenu sera explorée. Afin de clore le projet, les possibilités que la solution apportée à un groupe voulant poursuivre ou implémenter un projet semblable seront expliquées. L’équipe fera ces recommandations en fonction de l’expérience acquise lors du développement.
\end{document}
