\documentclass[rapport.tex]{subfiles}
\renewcommand{\abstractname}{REMERCIEMENT}
\renewcommand{\abstractnamefont}{\normalfont\Large\bfseries}
\begin{document}
\begin{abstract}
Cela fait maintenant 6 ans que 3 des 4 membres de notre équipe travaillons ensemble dans un but professionnellement commun. Nous avons vécu nos premiers travaux d’équipes lors de nos études collégiales et nous sommes très fiers de pouvoir dire que nous terminons une étape de nos vies en même temps. Nous avons rencontré le quatrième membre de cette fabuleuse équipe lors notre tout premier cours à l’École de Technologie Supérieure.
        \par
    S'étant suivis de session en session tout au long de notre Baccalauréat en génie des technologies de l’information, nous voulions absolument pouvoir concevoir notre projet de fin d’études entre nous. C’est grâce à l’aide de madame Lucie Caron et de monsieur Jean-Marc Robert nous avons pu réaliser ce projet de grande envergure entouré de coéquipier dont la synergie est plus que remarquable.
    \par
    Nos plus grands remerciements iront tout de même à monsieur Carlos Vasquez pour nous avoir supportés tout au long du projet. Afin de pouvoir effectuer le projet de fin d’études avec notre équipe prédéterminée, nous avions besoin de trouver un professeur du département de TI qui aurait été intéressé à nous supporter lors de la réalisation de notre projet. Notre premier choix a été de contacter rapidement monsieur Vasquez, car son cours d’application de technique numérique nous avait bien intéressés tous. Le fait d’avoir choisi monsieur Carlos Vasquez s’est vu être un très bon choix comme professeur-superviseur de PFE, car il nous a très bien supportés tout au long des étapes de conception de ce projet.
    \vspace{-\baselineskip}
\end{abstract}
\end{document}
