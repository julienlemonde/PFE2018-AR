\documentclass[rapport.tex]{subfiles}
\begin{document}
\chapter*{\uppercase{Planification}}
\addcontentsline{toc}{chapter}{PLANIFICATION}
\section*{Mise en place}
\addcontentsline{toc}{section}{Mise en place}
Étant donné que la totalité des membres de notre équipe avait de l’expérience dans le développement applicatif par la méthodologie Agile, nous sommes lancés dans ce projet avec cette façon de faire. Agile étant une méthodologie plutôt permissive dans son exécution, nous avons pu conserver seulement certaines parties de ce mode de travail. En effet, dès les premières rencontres d’équipe pour le choix du sujet du PFE, nous avions décidé que notre projet allait comporter 4 itérations avec présentation au client des avancements. Ici, le client a été remplacé par le professeur supervisant le projet. L’avantage de vouloir fonctionner avec la méthodologie Agile est que nous pouvions rectifier le tir si nos avancements de livrable en livrable de correspondait pas aux attentes du professeur superviseur.
\section*{Séparation des tâches}
addcontentsline{toc}{section}{Séparation des tâches}
Dès que notre sujet pour le projet a été choisi, nous nous sommes lancé dans un survol des tâches qui allaient à être réaliser et nous avons rapidement constaté que la plupart des livrables se séparaient en deux catégories complètement indépendantes l’une de l’autre. Afin de facilité et le développement et séparer les responsabilités et réduire les risques de ces sous-domaines, nous avons divisé les membres de l’équipe en deux sous-équipes pour le lancement du projet. Ses deux équipes s’occupaient des deux grandes familles de tâches qui entouraient les éléments les plus risqués du projet. Ces réalisations étaient risquées par le fait qu’il s’agit de grand inconnu pour l’équipe. Ces deux familles de tâches étaient : Le développement du programme AR; Le développement du programme interagissant avec le scanneur matériel. Le but de ses deux équipes était de partir chacun de leur côté pour le commencement afin d’éliminer les inconnus face à ses technologies dès le début du projet. Plus le projet avançait, plus il fallait s’intégrer en une seule équipe pour rejoindre ses deux idées et en faire une seule solution complète.
\section*{Analyse préliminaire}
\addcontentsline{toc}{section}{Analyse préliminaire}
Tout d’abord, aucun des membres de l’équipe n’avait des connaissances dans les technologies qui devaient être traitées dans les projets. Il a donc été nécessaire de faire une recherche préliminaire afin de gagner une compréhension nécessaire à la planification du projet. La recherche a permis de répondre à des enjeux sûrs qu’elles étaient les principales parties du projet? Quelles technologies étaient incluses implicitement aux ARKit et Structure Sensor? Quel environnement de développement devait être utilisé? Est-ce que les objectifs étaient réalistes dans les contraintes de temps et de ressources?
\par
Suite à cette recherche, le projet ayant été jugé comme réalisable a été séparé en plusieurs modules. C’est à dire des parties du projet qui peuvent être développées indépendamment et être assignées à différents étudiants. Ayant conscience de la possibilité de bloquants au cours du développement cette technique permet de ne pas limiter l’avancement du projet complet, dès qu’il y a un problème. Les différents modules identifiés sont: le développement du scanneur avec le Structure Sensor, le développement de l’environnement AR et le développement du squelette dans le modèle humain. Le dernier module a été identifié comme accessoire étant donné l’importance de l’enjeu. De plus, le développement des deux premiers modules et leur combinaison sont déjà importants en taille et complexité.
\par
Une fois les deux modules principaux identifiés, une séparation de l’équipe de développement a été faite. La séparation visait à permettre à chacun de se concentrer sur un type d’enjeu et de diminuer la grosseur des défis. De plus, cette méthode a permis de répondre à des contraintes qui entourait le projet et les technologies disponibles. En effet, un seul périphérique Structure Sensor était disponible. De plus, trois développeurs sur quatre avaient accès à un appareil Apple mobile.
\par
Une fois les sous-équipes identifiées, celles-ci ont dû séparer et lister les tâches à accomplir, les ressources documentaires disponibles autour de leur technologie et lister les risques que leurs technologies impliquent. Ces informations ont été ajoutées au plan de projet avec la description du projet et la mise en situation. Les tâches ont été par la suite séparées en trois jalons de développement.
\par
Pour le module de Structure Sensor, les trois jalons consistent en le transfert du projet Scanneur de Swift 2 à Swift 4 pour le premier. Pour le second, la cible était l’intégration dans la structure de l’application AR et le transfert du modèle dans l’application. Le troisième consistait en la conception d’une interface visuelle adéquate et une possibilité de faire un début de squelette sur le modèle humain.
\par
Pour le module de Structure Sensor, les trois jalons consistent en le transfert du projet Scanneur de Swift 2 à Swift 4 pour le premier. Pour le second, la cible était l’intégration dans la structure de l’application AR et le transfert du modèle dans l’application. Le troisième consistait en la conception d’une interface visuelle adéquate et une possibilité de faire un début de squelette sur le modèle humain.
\section*{Itérations}
\addcontentsline{toc}{section}{Itérations}
Chaque jalon représentant une partie du travail à faire contiendra toujours un certain nombre commun de regroupements de tâches. Pour commencer, avant d’entreprendre toute forme de travail, des recherches doivent être faites. Les éléments concernant les différentes méthodes de réalisation d’une tâche et les avantages et désavantages de chacune de celles-ci doivent être identifiés.
\par
Par la suite, une solution doit être sélectionnée en comparant les avantages et désavantages. Le contexte de développement de l’application doit être aussi pris en cause. Une fois la solution choisie, l’adaptation de la conception de la solution doit être faite pour le cadre du projet. Si la solution s’intègre au sein d’une autre solution, les documents de conceptions de la solution parentes sont adaptés. La conception est donc souvent itérative à travers les jalons et les tâches.
\par
Par la suite, la solution est intégrée dans l’application à l’aide de l’éditeur spécialisé XCode. La compilation et la mise en marche sont testées sur un appareil mobile Apple au fur à mesure du développement afin d’éviter les mauvaises surprises en fin de jalon. Une fois une tâche terminée, le cas d’utilisations principal de l’application est testé pour s’assurer qu’aucune autre fonctionnalité n’a été brisée.
\par
Le jalon est par la suite terminé avec une revue du travail fait et avec une présentation préparée par l’équipe pour le superviseur du projet. Dans ces revues on discute du travail fait, des bloquants et du travail à faire. Les artefacts de conception et documents de projets doivent aussi être mis à jour.
\section*{Clôture}
\addcontentsline{toc}{section}{Clôture}
Le développement applicatif terminer, certaines tâches doivent être faites afin de clore le projet. Parmi celles-ci est la revue des résultats obtenue par l’équipe de développement. Cette tâche a pour but d’évaluer l’état de l’application par rapport au résultat qui était attendu en début de projet. Dans le cas de différences entre les attentes et le produit final, un bilan de justification doit être tenu.
\par
Par la suite, l’équipe devra lister ces recommandations qui pourraient servir à une autre équipe voulant poursuivre le projet. Le contexte de projet de fin d’études permet en effet la poursuite d’un tel projet. Alors un transfert de connaissance doit être préparé afin de limiter les frictions des futures parties prenantes au projet. Les deux dernières tâches permettent la préparation de la présentation qui sera faite en fin de projet devant d’autres étudiants finissants.
\end{document}
