\documentclass[rapport.tex]{subfiles}
\begin{document}
\chapter*{\uppercase{CONCLUSION}}
\addcontentsline{toc}{chapter}{CONCLUSION}
Le développement d’application utilisant la réalité augmentée est encore très embryonnaire. Toutefois, la publication de certaines technologies comme le Structure Sensor de la compagnie Occipital et l’ARKit développé pour les appareils Apple amène des nouvelles possibilités. Chacune de ces technologies vient par contre avec ces inconvénients. La première est très peu adopté par les développeurs puisqu’elle demande l’achat d’un périphérique. Une faible communauté de développement limite donc les ressources possibles à la résolution de problème. La seconde introduit des limitations physiques qui peuvent être résolues qu’avec une amélioration physique de l’appareil. De plus, l’ARKit ne peut qu'utiliser des modèles préalablement ajoutés ou importés dans l’application.
\par
Le produit réalisé devait donc combiner les deux technologies pour réduire les risques et les faiblesses de chacune. Le niveau de production visé de l’application n'était qu’une preuve de concept. Un objectif de détection de modèle humain avait été ajouté à l’envergure du projet au début pour concrétiser la preuve de concept. Cet objectif n’était pas nécessaire au statut de succès du projet.
\par
L'application produite au cours de la session valide en effet une combinaison possible des deux technologies. En effet, l’utilisateur est en mesure de scanner un objet de l’environnement réel. Par la suite, celui-ci peut l’introduire dans une scène de réalité augmentée gérée par l’ARKit. Une manipulation simple du modèle est par la suite possible. La solution confirme les possibilités qu’apporte un périphérique comme le Structure Sensor. Elle valide en plus la facilité de développement introduite avec l’ARKit. L’objectif de détection de modèle humain a malheureusement dû être reporté à une poursuite possible du projet.
\par
L’application développée dans le cadre du projet n’est qu’une preuve de concept. Par contre, elle introduit des possibilités pour des domaines d’activités tels que le jeu vidéo et les commerces de détail.
\par
Le produit a été produit au moment où des limitations existent dans les deux technologies. Le ARKit est limité par une détection de profondeur peu performante. Alors, que le Structure Sensor manque grandement de ressource pour le développement d’application. Il est fort probable qu’Apple introduisent des appareils permettant de réaliser des meilleures captures d’environnement réel. De plus, dans le cas où l’adoption du périphérique Structure Sensor devient importante la compagnie Occipital risque d’introduire une documentation plus complète. Si ces prévisions s’avèrent véridiques, il serait recommandé de choisir une seule solution. Ce changement limitera alors les difficultés de développement.
\end{document}
