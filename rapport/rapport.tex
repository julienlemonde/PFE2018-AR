\documentclass[letterpaper,twoside,12pt,french]{report}
\usepackage[utf8]{inputenc}
\usepackage[T1]{fontenc}
\usepackage{babel}
\usepackage[top=2cm,bottom=3cm,left=2.5cm,right=2.5cm,asymmetric]{geometry}
\usepackage{hyperref}
\usepackage{hypcap}
\usepackage{titlesec}
\usepackage{ragged2e}
\usepackage{hyphenat}

\usepackage{abstract}
\renewcommand{\abstractname}{}
\renewcommand{\absnamepos}{empty}

\hypersetup{pdftex,colorlinks=true,allcolors=blue}
\newcommand{\reportTitle}{\uppercase{Détection de modèle humain\\Pour application de réalité augumentée}}
\newcommand{\reportAuthors}{Vincent \uppercase{Boiteau-Robert\\BOIV14029409}\\Marc-Antoine \uppercase{Hébert\\HEBM14019401}\\Julien \uppercase{Lemonde\\LEMJ20059208}\\Alexandre \uppercase{Malo\\MALA0702940}}

\newcommand{\abbrItem}[2]{\item[\uppercase{\textbf{#1}} -] #2}

\titleformat
{\chapter}
[display]
{\bfseries\Large}
{}
{0.5ex}
{
    \vspace{1ex}
}
{
    \vspace{-0.5ex}
}

\begin{document}
\setlength{\parindent}{0cm}\setlength{\parskip}{2ex plus 0.3ex minus 0.1ex}
\pagenumbering{Roman}
\begin{titlepage}
    \begin{center}
        \vfill
        \uppercase{Rapport technique de projet de fin d'études\\Présenté à l'école de technologie supérieure\\Dans le cadre du cours GTI795 Projet de fin d'études en TI}
        \vfill
        \textbf{\reportTitle}
        \vfill
        \reportAuthors
        \vfill
        \uppercase{Département de génie logiciel et TI}
        \vfill
        \textbf{Professeur-superviseur\\Carlos Vazquez}
        \vfill
        \uppercase{Montréal, 25 avril 2018\\Hiver 2018\\Université du Québec}
        \vfill
    \end{center}
\end{titlepage}
\cleardoublepage
\begin{abstract}
    \begin{center}
        \textbf{\uppercase{\Large{Résumé}}}
        \vspace{2ex}
        \\ \textbf{\reportTitle}
        \\ \vspace{1\baselineskip}
        \reportAuthors
        \\ \vspace{1\baselineskip}
        text résumé ici
    \end{center}
    \vspace{-\baselineskip}
\end{abstract}
\cleardoublepage
\tableofcontents
\cleardoublepage
\listoftables
\cleardoublepage
\listoffigures
\cleardoublepage
\chapter*{\uppercase{Liste des abbréviations}}
\begin{enumerate}
        \abbrItem{ÉTS}{École de Technologie Supérieure}
\end{enumerate}
\cleardoublepage
\pagenumbering{arabic}
\chapter*{\uppercase{Introduction}}
\addcontentsline{toc}{chapter}{INTRODUCTION}
\section*{Problématique}
Les avancés technologiques des caméras sur les appareils mobiles ont concrétisés la possibilité
de la réalité augmentée au cours des dernières années. Le développement est maintenant plus
démocratisé avec la création d'environnement comme le ARKit de Apple (Statt, N. 2017.).
Toutefois, ces technologies permettent l'interaction avec des modèles généré de façon
synthétique. Un défi qui n'a pas encore de solution reconnue et acceptée par la communauté est
l'usage de modèles provenant de l'environnement de l'utilisateur dans la réalité augmentée.
\par
Des technologies comme le Structure Sensor permettent la modélisation d'objets dans un
environnement réel afin de créer des modèles 3D numériques (Molitch-Hou, M. 2016.). Par contre,
celui-ci étant une nouvelle technologie et ayant une clientèle assez niche a peu de solutions
pour interagir avec les autres technologies de l'appareil.
\par
Ainsi la première problématique qui souhaite être adressé par ce projet est le développement
d'une technique d'interaction entre les deux technologies. Ce genre d'avancer permettrait donc
de non seulement à l'utilisateur d'ajouter des objets dans son environnement. Mais aussi
amener des modifications à ce qui l'entoure.
\section*{Objectifs}
Comme décrit dans la section précédente, le principal objectif du projet est la création d'une
interface entre deux technologies mobiles, soit un scanner d'environnement 3D, le Structure
Sensor (Molitch-Hou, M. 2016.), et la trousse de développement de réalité augmentée d'Apple
(Statt, N. 2017.). Autrement dit, l'équipe s'attend en fin de session d'avoir une application mobile
permettant la prise de modèle à l'aide du senseur et l'ajout d'un modèle extrait sur une surface
plane. L'application devra permettre cette chaîne d'action avec une expérience utilisateur
agréable et naturelle à l'utilisateur.
\par
L'application décrite ci-haut représente les attentes pessimistes par rapport aux résultats finaux.
Toutefois, une telle application apporte un éventail de possibilités sur la manipulation de
l'environnement augmentée. Par exemple, une des retombées envisageables et
commercialisable est la numérisation de catalogue de magasin comme Ikea. L'usager peut
donc ajouter un item dans une pièce de sa maison. Si plusieurs couleurs sont disponibles pour
un modèle, chacune des variantes seront disponibles à la modélisation.
\par
Une autre ouverture technologique, qu'une telle application apporterait, est dans un contexte
d'animation 3D. En effet, la création d'un modèle permettrait l'ajout d'un squelette dans celui-ci.
Une fois le squelette créé l'animation du modèle pourrait être fait. Cette possibilité est
particulièrement intéressante pour l'équipe en charge du projet. Dans l'éventualité où la vision
de base du projet serait complétée avant la fin de la session d'Hiver 2018, l'équipe compte se
concentrer sur la création de ce squelette et possiblement la manipulation de celui-ci.
\section*{Hypothèses}
\chapter*{\uppercase{Revue de la documentation}}
\addcontentsline{toc}{chapter}{REVUE DE LA DOCUMENTATION}
revue text
\chapter*{\uppercase{Méthodologie de travail}}
\addcontentsline{toc}{chapter}{MÉTHODOLOGIE DE TRAVAIL}
méthodologie text
\chapter*{\uppercase{Processus de conception}}
\addcontentsline{toc}{chapter}{PROCESSUS DE CONCEPTION}
processus text
\chapter*{\uppercase{Discussion}}
\addcontentsline{toc}{chapter}{DISCUSSION}
discussion text
\chapter*{\uppercase{Conclusion}}
\addcontentsline{toc}{chapter}{CONCLUSION}
conclusion text
\end{document}
